\documentclass[11pt,a4paper]{moderncv}

\usepackage{fontspec}
\usepackage[brazil]{babel}
\usepackage[scale=0.8, top=1cm, bottom=2cm]{geometry}

\moderncvstyle{classic}
\moderncvcolor{red}

\firstname{Cauê}
\familyname{Baasch de Souza}

\address{}{Florianópolis, SC}
\mobile{(48) 99634-5456}
\email{cauebs@pm.me}

\title{Curriculum Vitae}

\begin{document}
\makecvtitle{}
\pagestyle{empty}

\section{Educação}

\cventry
    {2015--2016}
    {Bacharelado em Engenharia Mecânica (Incompleto)}
    {}{}{}
    {Universidade Federal de Santa Catarina}

\cventry
    {2016--}
    {Bacharelado em Ciências da Computação}
    {}{}{}
    {Universidade Federal de Santa Catarina}

\section{Competências}
    \subsection{Programação}
        \cvitem{Avançado}{Python, Rust}
        \cvitem{Intermediário}{C, C++, Haskell, Elm, MicroPython}
        \cvitem{Básico}{TypeScript, Arduino}

    \subsection{Ferramentas}
        \cvitem{Avançado}{Poetry}
        \cvitem{Intermediário}{Git, Cargo, Docker, Podman}
        \cvitem{Básico}{Ninja, Meson, GNU Make}

    \subsection{Ambientes de desenvolvimento}
        \cvitem{Avançado}{Visual Studio Code}
        \cvitem{Intermediário}{Atom, Vim, Jupyter}
        \cvitem{Básico}{PyCharm}

    \subsection{Outros}
        \cvitem{}{Bancos de dados relacionais (SQLite, PostgreSQL).}
        \cvitem{}{Bancos de dados colunares (Cassandra).}
        \cvitem{}{Linux (Arch Linux).}

    \subsection{Conhecimentos e interesses}
        \cvitem{}{Projeto e implementação de linguagens.}
        \cvitem{}{Análise estática de código.}
        \cvitem{}{Desenvolvimento e análise de algoritmos.}
        \cvitem{}{Benchmarking e otimização.}
        \cvitem{}{Desenvolvimento de ferramentas e automatização de processos.}
        \cvitem{}{Sistemas de informações geográficas (GIS).}
        \cvitem{}{Sistemas embarcados.}
        \cvitem{}{Ensino de programação (e.g. Python, Rust, MicroPython.)}

    \subsection{Línguas}
        \cvitem{Inglês}{Fluente para leitura, escrita e conversação.}
        \cvitem{Português}{Fluente para leitura, escrita e conversação.}
        \cvitem{Francês}{Nível básico.}

\newpage

\section{Experiências profissionais}
    \cventry{2016--2017}{Bolsista}
    {Laboratório de Transportes e Logística (LabTrans) - UFSC}
    {}{}{
        \begin{itemize}
            \item Python.
            \item Qt, Cassandra.
            \item Pandas, Geopandas, Shapely, Matplotlib, Folium, PostGIS.
        \end{itemize}
    }

    \cventry{2018}{Autônomo}
    {Engie}
    {}{}{Desenvolvimento de um crawler.}

    \cventry{2017--}{Coordenador, membro fundador}
    {Caravela HackerClub}
    {(coletivo estudantil)}{}{}

    \cventry{2018--2019}{Tesoureiro}
    {Centro Acadêmico Livre da Computação (CALICO)}
    {}{}{}

    \cventry{2019--}{Vice-presidente}
    {Centro Acadêmico Livre da Computação (CALICO)}
    {}{}{}

    \cventry{2019--}{Coordenador Geral}
    {Diretório Central dos Estudantes (DCE) - Luís Travassos}
    {}{}{}

    \cventry{2020}{Estagiário de desenvolvimento}
    {Expertise Solutions}
    {}{}{Desenvolvimento de port nativo do framework EFL para Windows (C/C++).}

    \subsection{Ensino}
        \cvitem{2017--2019}{Minicurso de Python @ UFSC (voluntário) (~80 horas-aula)}
        \cvitem{2018}{Oficina de Rust @ UFSC (voluntário) (4 horas-aula)}
        \cvitem{}{Oficina de Rust @ SECCOM - UFSC (2 horas-aula)}

\section{Experiência acadêmica}

    \cventry{2019}{Departamento de Informática e Estatística - UFSC}{Bolsista de pesquisa}
    {}{}{
        \begin{itemize}
            \item Fusão de dados em IoT.
            \item Microsserviços para integração de protocolos de comunicação em IoT.
            \item Arduino, Zigbee, MicroPython, sockets.
        \end{itemize}
    }

\section{Eventos}
    \subsection{Organização}
        \cvitem{2018}{32º Python Floripa @ UFSC}
        \cvitem{}{33º Python Floripa @ UFSC}
        \cvitem{}{1º Rust Floripa @ CIASC}
        \cvitem{}{2º Rust Floripa @ CIASC}
        \cvitem{}{Todos a Bordo \#1 (Caravela Hackerclub)}
        \cvitem{}{Todos a Bordo \#2 (Caravela Hackerclub)}
        \cvitem{}{37º Python Floripa @ UFSC}
        \cvitem{2019}{Semana Acadêmica de Ciência da Computaçâo e Sistemas de Informação da UFSC (SECCOM)}

    \subsection{Apresentação}
        \cvitem{2017}{Estruturas de dados no CPython @ 25º Python Floripa (SENAI)}
        \cvitem{2018}{Criando e empacotando aplicações CLI em Rust @ Native Floripa}
        \cvitem{2019}{MicroPython: o Arduino dos Pythonistas @ 36º Python Floripa (Celta)}
        \cvitem{}{Programação de Linguagens (projeto de linguagens de programação) @ Todos a Bordo \#4 (Caravela Hackerclub)}
        \cvitem{}{Com licença eu tenho direitos (propriedade intelectual e software livre) @ Todos a Bordo \#5 (Caravela Hackerclub)}


\section{Repositórios pessoais}
\cvitem{}{\url{https://github.com/cauebs}}

\thispagestyle{plain}

\end{document}
